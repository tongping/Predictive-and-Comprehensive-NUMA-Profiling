\section{Limitation}
\label{sec:discussion}
\NP{} bases on compiler-based instrumentation to capture memory accesses. Therefore, it shares the same shortcomings and strengths of all compiler-based instrumentation. On the one side, \NP{} can perform static analysis to reduce unnecessary memory accesses, such as accesses of stack variables. \NP{} typically achieves much better performance than binary-based instrumentation tools, such as Numaprof~\cite{valat:2018:numaprof}. On the other side, \NP{} requires the re-compilation (and the availability of the source code), and will miss memory accesses without the instrumentation. That is, it can not detect NUMA issues caused by non-instrumented components (e.g., libraries), suffering from false negatives. However, most issues should only occur in applications, but not libraries. 