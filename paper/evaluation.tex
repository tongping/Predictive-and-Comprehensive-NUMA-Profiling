\section{Experimental Evaluation}
\label{sec:evaluation}

\todo{Can we differentiate the reason of remote accesses. First, we should differentiate false and true sharing. For true sharing, we should at least differentiate read-mostly and others, since read-mostly are easier to achieve performance speedup. }
\subsection{Effectiveness}
We will evaluate this work against two existing work:

%https://github.com/memtt/numaprof
%

\todo{We should detect more issues in real applications, such as Apache and MySQL and MemCached} 

%\renewcommand{\arraystretch}{1.5}
\begin{table*}[tp]
\footnotesize
%	\setlength{\tabcolsep}{0.3em}
  \centering
    \begin{tabular}{|l|r|r|r|r|r|r|r|r|}
    \hline
    \cline{1-7}
    Benchmark&Issue&Strategy&New&Source Code&Improvement&Final Improvement\\ \hline
    bodytrack&page level false sharing&interleaved pages&\checkmark&FlexImageStore.h:146&0&0 \\
    &page\&cache level false sharing&interleaved pages\&pedding&\checkmark&ParticleFilter.h:205&0&0\\
    &thread migration&round robin thread binding&\checkmark&&0&0\\ \hline
    canneal&thread migration&round robin thread binding&\checkmark&&0&0\\ \hline
    dedup&memory access imbalance&thread clustering&\checkmark&&0&0\\ 
    &thread stage load imbalance&adjust thread numbers&\checkmark&&0&0\\ 
    &thread migration&round robin thread binding&\checkmark&&0&0\\\hline
    facesim&page level false sharing&interleaved pages&\checkmark&taskQ.c:219&0&0\\ 
    &thread migration&round robin thread binding&\checkmark&&0&0\\
    \hline
    ferret&page level false sharing&interleaved pages&\checkmark&dataset.c:224&0&0\\
    &memory access imbalance&thread clustering&\checkmark&&0&0\\ 
    &thread stage load imbalance&adjust thread numbers&\checkmark&&0&0\\ 
    &thread migration&round robin thread binding&\checkmark&&0&0\\
    \hline
    fluidanimate&page level false sharing&interleaved pages&\checkmark&pthreads.cpp:294&0&0\\
    &page level false sharing&interleaved pages&\checkmark&pthreads.cpp:292&0&0\\
    &thread migration&round robin thread binding&\checkmark&&0&0\\
    \hline
    streamcluster&cache level false sharing&adjust CACHE\_LINE to 64 bytes&&streamcluster.cpp:984&0&0\\ 
    &page level false sharing&duplicate over nodes&\checkmark&streamcluster.cpp:1845&0&0\\
    &page level false sharing&interleaved pages&\checkmark&streamcluster.cpp:1906&0&0\\
    &thread migration&round robin thread binding&\checkmark&&0&0\\
    \hline
    vips&page level false sharing&interleaved pages&\checkmark&memory.c:156&0&0\\ 
    &thread migration&round robin thread binding&\checkmark&&0&0\\
    \hline
    x264&thread migration&round robin thread binding&\checkmark&&0&0\\
    \hline
    \hline  
    lulesh&page level false sharing&memory partition&&lulesh.cc:543-545&0&0\\ 
    &page level false sharing&memory partition&\checkmark&lulesh.cc:2251-2262&0&0\\
    &page level false sharing&memory partition&&lulesh.cc:1029-1031&0&0\\
    &page level false sharing&memory partition&\checkmark&lulesh.cc:1096&0&0\\
    &thread migration&round robin thread binding&\checkmark&&0&0\\
    \hline 
    AMG2006&page level false sharing&interleaved pages&\checkmark&par\_relax.c:1631&0&0\\ 
    &page level false sharing&block-wised interleaved&&par\_rap.c:1385,1286&0&0\\
    &thread migration&round robin thread binding&\checkmark&&0&0\\
    \hline 
    UMT2013&thread migration&round robin thread binding&\checkmark&&0&0\\
    \hline 
    \end{tabular}
  \caption{NUMA issues in PARSEC benchmarks. \label{tab:numa_issues}}
\end{table*}


\subsection{Case Studies}
\label{sec:casestudies}

\subsubsection{Fixing Excessive Number of Remote Accesses}

\subsubsection{Fixing Thread Migration} 

\subsubsection{Fixing Load Imbalance}
\subsection{Performance Overhead}

\subsection{Memory Overhead}

\subsection{Sensitive To Inputs}
We further confirms that most performance issues can be detected with different inputs or commands. 

We may try Apache and MySQL to check whether the detected issues are changed with different inputs or different commands. 

Then we may reach a conclusion that there is no need to detect them during the deploy environment. 
