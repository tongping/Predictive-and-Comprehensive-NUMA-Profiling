\section{Experimental Evaluation}
\label{sec:evaluation}
%\begin{comment}

This section aims to answer the following research questions: 

\begin{itemize}
\item \textbf{Effectiveness:} Whether \NP{} could detect more performance issues than existing NUMA-profilers? (Section~\ref{effectiveness}) How helpful of \NP{}'s detection report? (Section~\ref{sec:casestudies})
\item \textbf{Performance:} How much performance overhead is imposed by \NP{}'s detection, comparing to the state-of-the-art tool? (Section~\ref{sec:performance}) 
\item \textbf{Memory Overhead:} What is the memory overhead of \NP{}? (Section~\ref{sec:memory})
\item \textbf{Architecture In-dependence:} Whether \NP{} could detect similar issues when running on a non-NUMA architecture? (Section~\ref{sec:archindependent})	
\end{itemize}

%\end{comment}

\textbf{Experimental Platform:}  \NP{} was evaluated on a machine with 8 nodes and 128 physical cores in total, except in Section~\ref{sec:archindependent}. This machine is installed with 512GB memory. Any two nodes in this machine are less than or equal to 3 hops, where the latency of two hops and three hops is 2.1 and 3.1 separately, while the local latency is 1.0. The OS for this machine is Linux Debian 10 and the compiler is GCC-8.3.0. The hyperthreading was turned off for the evaluation. 


%\todo{Can we differentiate the reason of remote accesses. First, we should differentiate false and true sharing. For true sharing, we should at least differentiate read-mostly and others, since read-mostly are easier to achieve performance speedup. }
\subsection{Effectiveness}
\label{effectiveness}
%%\renewcommand{\arraystretch}{1.5}
\begin{table}[!htp]
% \footnotesize
% \setlength{\tabcolsep}{0.3em}
  \centering
    \begin{tabular}{|l|r|}
    \hline
    \multirow{1}{*}{Issue Categories}&Score \\ \hline
    \PS&1500\\ \hline
    \FS&100\\ \hline
    \TS&100\\ \hline
    \TM&150\\ \hline
    \end{tabular}
  \caption{Score threshold for different categories of issues.
  \label{tab:memory_consumption}}
\end{table}
%\renewcommand{\arraystretch}{1.5}
\begin{table*}[tp]
\footnotesize
%	\setlength{\tabcolsep}{0.3em}
  \centering
    \begin{tabular}{|l|r|r|r|r|r|r|r|r|}
    \hline
    \cline{1-7}
    Benchmark&Issue&Strategy&New&Source Code&Improvement&Final Improvement\\ \hline
    bodytrack&page level false sharing&interleaved pages&\checkmark&FlexImageStore.h:146&0&0 \\
    &page\&cache level false sharing&interleaved pages\&pedding&\checkmark&ParticleFilter.h:205&0&0\\
    &thread migration&round robin thread binding&\checkmark&&0&0\\ \hline
    canneal&thread migration&round robin thread binding&\checkmark&&0&0\\ \hline
    dedup&memory access imbalance&thread clustering&\checkmark&&0&0\\ 
    &thread stage load imbalance&adjust thread numbers&\checkmark&&0&0\\ 
    &thread migration&round robin thread binding&\checkmark&&0&0\\\hline
    facesim&page level false sharing&interleaved pages&\checkmark&taskQ.c:219&0&0\\ 
    &thread migration&round robin thread binding&\checkmark&&0&0\\
    \hline
    ferret&page level false sharing&interleaved pages&\checkmark&dataset.c:224&0&0\\
    &memory access imbalance&thread clustering&\checkmark&&0&0\\ 
    &thread stage load imbalance&adjust thread numbers&\checkmark&&0&0\\ 
    &thread migration&round robin thread binding&\checkmark&&0&0\\
    \hline
    fluidanimate&page level false sharing&interleaved pages&\checkmark&pthreads.cpp:294&0&0\\
    &page level false sharing&interleaved pages&\checkmark&pthreads.cpp:292&0&0\\
    &thread migration&round robin thread binding&\checkmark&&0&0\\
    \hline
    streamcluster&cache level false sharing&adjust CACHE\_LINE to 64 bytes&&streamcluster.cpp:984&0&0\\ 
    &page level false sharing&duplicate over nodes&\checkmark&streamcluster.cpp:1845&0&0\\
    &page level false sharing&interleaved pages&\checkmark&streamcluster.cpp:1906&0&0\\
    &thread migration&round robin thread binding&\checkmark&&0&0\\
    \hline
    vips&page level false sharing&interleaved pages&\checkmark&memory.c:156&0&0\\ 
    &thread migration&round robin thread binding&\checkmark&&0&0\\
    \hline
    x264&thread migration&round robin thread binding&\checkmark&&0&0\\
    \hline
    \hline  
    lulesh&page level false sharing&memory partition&&lulesh.cc:543-545&0&0\\ 
    &page level false sharing&memory partition&\checkmark&lulesh.cc:2251-2262&0&0\\
    &page level false sharing&memory partition&&lulesh.cc:1029-1031&0&0\\
    &page level false sharing&memory partition&\checkmark&lulesh.cc:1096&0&0\\
    &thread migration&round robin thread binding&\checkmark&&0&0\\
    \hline 
    AMG2006&page level false sharing&interleaved pages&\checkmark&par\_relax.c:1631&0&0\\ 
    &page level false sharing&block-wised interleaved&&par\_rap.c:1385,1286&0&0\\
    &thread migration&round robin thread binding&\checkmark&&0&0\\
    \hline 
    UMT2013&thread migration&round robin thread binding&\checkmark&&0&0\\
    \hline 
    \end{tabular}
  \caption{NUMA issues in PARSEC benchmarks. \label{tab:numa_issues}}
\end{table*}


We evaluated \NP{} on multiple HPC applications (e.g.,  AMG2006~\cite{AMG2006}, lulesh~\cite{LULESH}, and UMT2013~\cite{UMT2013}) and a widely-used multithreaded application benchmark suite --- PARSEC~\cite{parsec}.  Applications with NUMA performance issues are listed in Table~\ref{tab:numa_issues}.  The performance improvement after fixing all issues is listed in ``Improve'' column, with the average of 10 runs, where all specific issues are listed afterwards. For each  issue, the table listed the type of issue and the corresponding score, the allocation site, and the fix strategy. Note that the table only shows cases with page sharing score larger than 1500 (if without cache false/true sharing), false/true sharing score larger than 1, and thread migration score larger than 150. Further, the performance improvement of each specific issue is listed as well. We also present multiple cases studies that show how \NP{}'s report is able to assist bug fixes in Section~\ref{sec:casestudies}.   

Overall, we have the following observations. First, it reports no false positives by only reporting scores larger than a threshold. Second, \NP{} 
detects more performance issues than the combination of all existing NUMA profilers~\cite{Intel:VTune, Memphis, Lachaize:2012:MMP:2342821.2342826, XuNuma, NumaMMA, 7847070, diener2015characterizing, valat:2018:numaprof}. The performance issues that cannot be detected by existing NUMA profilers are highlighted with a check mark in the last column of the table, although some can be detected by specific tools, such as cache false/true sharing issues~\cite{Sheriff, Predator, Cheetah, DBLP:conf/ppopp/ChabbiWL18, helm2019perfmemplus}. This comparison with existing NUMA profilers is based on the methodology. Existing NUMA profilers cannot separate false or true sharing with normal remote accesses, and cannot detect thread migration and load imbalance issues.
%Note that false sharing or true sharing issues, as well as thread imbalance issues, can be detected using specific tools, but cannot be detected by existing NUMA profilers. That is the reason why they are still labelled as ``New'' in the table. 

When comparing to a specific profiler, \NP{} also has better results even on detecting remote accesses. For lulesh, HPCToolkit detects issues of \# 4 ~\cite{XuNuma}, while \NP{} detects three more issues (\# 3, 5, 7). Fixing these issues improves the performance by up to 504\% (with the threads binding). Multiple reasons may contribute to this big difference. 
 %Different detection effectiveness between \NP{} and HPCToolkit can be caused by multiple reasons. 
 First, \NP{}'s predictive method detects some issues that are not occurred in the current scheduling and the current hardware, while HPCToolkit has no such capabilities. Second, HPCToolkit requires to bind threads to nodes, which may miss remote accesses caused by its specific binding. 
 %, while \NP{} overcomes this issue by focusing on thread-relationship. 
Third, \NP{}'s fine-grained profiling provides a better effectiveness than a coarse-grained profiler like HPCToolkit. 
\NP{} may have false negatives caused by its instrumentation. \NP{} cannot detect an issue of UMT2013 reported by HPCToolkit~\cite{XuNuma}. The basic reason is that \NP{} cannot instrument Fortran code. \NP{}'s limitations are further discussed in Section~\ref{sec:casestudies}.


%Unfortunately, the most current LLVM\cite{LLVMFLANG} does not support to generate LLVM IR for the Fortran source code and thus is unable to do instrumentation. So \NP can not detect this issues.


\subsection{Case Studies}
\label{sec:casestudies}
In this section, multiple case studies are shown how programmers could fix performance issues based on the report. 
\lstset{ %
backgroundcolor=\color{white},      % choose the background color
basicstyle=\footnotesize\ttfamily,  % size of fonts used for the code
columns=fullflexible,
tabsize=4,
breaklines=true,               % automatic line breaking only at whitespace
captionpos=b,                  % sets the caption-position to bottom
commentstyle=\color{mygreen},  % comment style
escapeinside={\%*}{*)},        % if you want to add LaTeX within your code
keywordstyle=\color{blue},     % keyword style
stringstyle=\color{mymauve}\ttfamily,  % string literal style
frame=single,
rulesepcolor=\color{red!20!green!20!blue!20},
% identifierstyle=\color{red},
language=c++,
}


% \todo{Xin: we don't need to discuss every case here. we only need to show one or two examples , such as lulesh, and fluidanimate. Better to choose those cases with significant performance improvement. For each case, we should show what \NP{} reports. How to fix the issue based on report? }
\subsubsection{Remote Accesses}

% What type of score that \NP{} will report? 
% How \NP{} could provide sufficient information for different types of bugs? 
% For instance, why we should use replication based on report? 
% Why we should use the block interleaved method? Why we should use padding? Why we should just use interleaved allocation? 

For remote accesses, \NP{}  not only reports remote access scores, indicating the seriousness of the corresponding issue, but also provides additional information to assist bug fixes. Remote accesses can be fixed with different strategies, such as padding (false sharing), block-wise interleaving, duplication, and page interleaving. 

\begin{lstlisting}[caption={Remote access issue of lulesh },label={blockinterleave},captionpos=b]
Allocation Site: lulesh.cc:2251
Remote score:  4496
False sharing score:  26
True Sharing score:   0.00
Pages accessed by threads:
    0--8, 8--16, 16--23, 23--31 ......
\end{lstlisting}

%\todo{xin updated. I didn't find lulesh.cc:176 has an issue. I also didn't find the score. Make sure the score is the same as Table 1 and Table 3}. 
\NP{} provides a data-centric analysis, as existing work~\cite{XuNuma}. That is, it always attributes performance issues to its allocation callsite. \NP{} also shows the seriousness with its remote access score.

%First of all, a high page score like Listing~\ref{duplicate},~\ref{blockinterleave} means objects created from the malloc site are highly shared by multiple threads, so it is under a high potential to cause massive remote access. In this case, 
\NP{} further reports more specific information to guide the fix. As shown in  Listing~\ref{blockinterleave}, \NP{} further reports each page that are accessed by which threads. Based on this information,  block-wise interleave is a better strategy for the fix, which achieves a better performance result. However, for Issue 17 or 19 of \texttt{luresh}, there is no such access pattern. Therefore, these bugs can be fixed with the normal page interleave method. 

\begin{lstlisting}[caption={Remote access issue of streamcluster},label={duplicate},captionpos=b]
Allocation site:streamcluster.cpp:1845
Remote score:   7169
False sharing score:  0.00
True Sharing score:   0.00
Continuous reads after the last write:   2443582804
\end{lstlisting}

%\todo{Sorry, do not have time to do this. Better to check if we don't use duplications, how much performance we can achieve for this example. But this is in lower priority} 
Listing~\ref{duplicate} shows another example of remote accesses. For this issue (\# 24), a huge number of continuous reads (2330M) were detected after the last write. Based on such a report, the object can be duplicated to different physical nodes, which improves the performance by 158\%, which achieves significantly better performance than page interleave. 

For cache coherency issues,  \NP{} differentiates them from normal remote accesses, and further differentiates false sharing from true sharing. Given the report, programmers could utilize the padding to eliminate false sharing issues. As shown in Table~\ref{sec:evaluation}, many issues have false sharing issues (e.g., \#6, \#8, \#12, \#20, \#23). Fixing them with the padding could easily boost the performance. However, we may simply utilize the page interleave to solve true sharing issues. 

%However a further interleaved strategy could give more benefits after padding as showed in Table~\ref{sec:evaluation} for Issue[5,7,11,19,22].But true sharing is very hard to be eliminated, so a simple interleaved strategy is applied for Issue[18] in Table~\ref{sec:evaluation} to reduce resource contention for NUMA architecture. 

\subsubsection{Thread Migration} 
%\todo{xin new writing
When an application has frequent thread migrations, it may introduce excessive thread migrations. For such issues, the fix strategy is to bind threads to nodes. Typically, there are two strategies: round robin and packed binding. Round robin is to bind continuous threads to different nodes one by one,  ensuring that different nodes have a similar number of threads. Packed binding is to bind multiple threads to the first node, typically the same as the number of hardware cores in one node, and then to another node afterwards. Based on our observation, round robin typically achieves a better performance than packed binding, which is the default binding policy for our evaluations in Table~\ref{tab:numa_issues}. Thread binding itself achieves the performance improvement by up to 418\% (e.g., \texttt{fluidanimate}), which indicates the importance for some applications. 

%Listing~\ref{blockinterleave} 

%For \NP{}  to help users to pick a better one. For example, in , packed binding could give further benefits combined with block-wise interleaved, since it not only could eliminate contention to the memory controller, but also could make most remote memory access turned into local access. On the other hand, if no such memory access pattern is found, users could just pick anyone. In our experiments, round-robin is picked by default.
%}
%\todo{Thread migration can be fixed with thread binding. There are two ways of binding: packed binding or round-robin binding? Can \NP{} tells us what type of binding will provide better performance? }

\subsubsection{Load Imbalance}

%For load imbalance issues, 
\NP{} not only reports the existence of such issues, but also suggests an assignment based on the number of sampled memory accesses. Programmers could fix them based on the suggestion. 
%In all evaluated applications, only two applications, \texttt{dedup} and \texttt{ferret}, have load imbalance issues, where they all employ the producer-consumer model and have multiple types of threads. 

For \texttt{dedup}, \NP{} reports that memory accesses of anchor, chunk, and compress threads have a proportion of 92.2:0.33:3.43, when all libraries are instrumented. That is, the portion of the chunk and compress threads is around 1 to 10. By checking the code, we understand that \texttt{dedup} has multiple stages, where the anchor is the previous stage of the chunk, and the chunk is the predecessor of the compress. Threads of a previous stage will store results into multiple queues, which will be consumed by threads of its next stage. Based on a common sense that many threads competing for the same queue may actually introduce high contention. Therefore, the fix will simply set the number of chunk threads to be 2. Based on this, we further set the number of compress threads to be 18, and the number of anchor to be 76. The corresponding queues are 18:2:2:4. With this setting, \texttt{dedup}'s performance is improved by 116\%. We further compare its performance with the suggested assignment of another existing work--SyncPerf~\cite{SyncPerf}. SyncPerf assumes that different types of threads should have the same waiting time. SyncPerf proposes the best assignment should be 24:24:48, which could only improve the performance by 105\%. 
%That is, \NP{}'s suggestion is much better than SyncPerf's suggestion for load imbalance issue. 

In another example of \texttt{ferret}, \NP{} suggests a proportion of $3.3 :1.9 :47.4 :75.3$ for its four types of threads. With this suggestion, we are configuring the threads to be $4 : 2 : 47 : 75$. With this assignment, \texttt{ferret}'s performance increases by 206\% compared with the original version. In contrast, SyncPerf suggests an assignment of $1:1:2:124$
. However, following such an assignment  actually degrades the performance by 354\% instead. 

\subsection{Performance Overhead}
\label{sec:performance}
\begin{figure}[!h]
    \centering
    \includegraphics[width=3.2in]{paper/figures/performance.pdf}
    \caption{Performance overhead of \NP{} and others.\label{fig:performance}}  
\end{figure}
%\todo{xin new writing

We also evaluated the performance of \NP{} on PARSEC applications, and the performance results are shown in Figure~\ref{fig:performance}. On average, \NP{}'s overhead is around 585\%, which is orders-of-magnitude smaller than the state-of-the-art fine-grained profiler --- NUMAPROF~\cite{valat:2018:numaprof}. In contrast, NUMAPROF's overhead runs $316\times$ slower than the original one. \NP{} is designed carefully to avoid such high overhead, as discussed in Section~\ref{sec:implementation}. Also, \NP{}'s compiler-instrumentation also helps reduce some overhead by excluding memory accesses on stack variables. 

There are some exceptions. Two applications impose more than $10\times$ overhead, including Swaption and x264. Based on our investigation, the instrumentation with an empty function imposes more than $5\times$ overhead. The reason is that they have significantly more  memory accesses compared with other applications like blackscholes. Based on our investigation, swaption has more than $250\times$ memory accesses than  blackscholes in a time unit. Applications with low overhead can be caused by not instrumenting libraries, which is typically not the source of NUMA performance issues. 

%Besides, \NP{} has to do lots of computation inside the intercepting function which could make memory access go up to multiple times. And if a target application contains any NUMA issues, the issue also could arise in \NP{}. Based on this situation, we believe \NP{} did it very well to achieve a good performance. 

\subsection{Memory Overhead}
\label{sec:memory}
%\renewcommand{\arraystretch}{1.5}
\begin{table}[!htp]
% \footnotesize
% \setlength{\tabcolsep}{0.3em}
  \centering
    \begin{tabular}{|l|r|r|r|}
    \hline
    \multirow{2}{*}{Apps}&
    \multicolumn{3}{c|}{Memory Usage (MB)}\\
    \cline{2-4}
    &Glibc&\NP&NUMAPROF \\ \hline
    \hline
    blackscholes&617&689&685\\ \hline
    bodytrack&36&139&260\\ \hline
    canneal&887&1476&2383\\ \hline
    dedup&917&1806&2388\\ \hline
    facesim&2638&2826&3005\\ \hline
    ferret&160&301&445\\ \hline
    fluidanimate&470&667&753\\ \hline
    raytrace&1287&1610&2089\\ \hline
    streamcluster&112&216&928\\ \hline
    swaptions&28&67&255\\ \hline
    vips&226&283&463\\ \hline
    x264&2861&3039&3108\\ \hline \hline  
    Total&{\bf 10238}&{\bf 13120}&{\bf 16762}\cr\hline
    \end{tabular}
  \caption{Memory consumption of different profilers. \label{tab:memory_consumption}}
  \vspace{-0.2in}
\end{table}

We further evaluated \NP{}'s memory overhead with PARSEC applications. The results are shown in Table~\ref{tab:memory_consumption}. In total, \NP{}'s memory overhead is around 28\%, which is much smaller than the state-of-the-art fine-grained profiler --- NUMAPROF~\cite{valat:2018:numaprof}. \NP{}'s memory overhead is mainly coming from the following resources. First, \NP{} records the detailed information in page-level and cache-level, so that we could provide detailed information to assist bug fixes. Second, \NP{} also stores allocation callsites for every object in order to attribute performance issues back to the data. 

We  notice that some applications have a larger percentage of memory overhead, such as \texttt{streamcluster}. For it, a large object has very serious NUMA issues. Therefore, recording page and cache level detailed information contributes to the major memory overhead. However, overall, \NP{}'s memory overhead is totally acceptable, since it provides much more helpful information to assist bug fixes. 

%and thread based levels, especially \NP{} also provides lots of help information to help users fix the issues.
%Because \NP{} has to use lots of memory to record variety memory access patterns in all levels for potential issues. That is why some applications like bodytrack, dedup and streamcluster could get 2 times memory overheads.In these applications, they got some very huge objects with very serious NUMA issues ,like streamcluster contained an object that could occupy 100MB.To track the access pattern and provide detailed help information, doubled or more memory overheads are very reasonable.So to reduce memory overheads, \NP{} applied lots of mechanisms to avoid some objects if they look like not suspicious.That is why some applications like blackscholes, facesim and x264 bring very tiny memory overheads.Overall, \NP{} occupied less memory compared with NUMAPROF, but provided more information.

%\todo{Adding some explanation for memory utilization}. 

\subsection{Architecture Sensitiveness}
\label{sec:archindependent}

We further confirm whether \NP{} is able to detect similar performance issues when running on a non-NUMA or UMA machine. We further performed the experiments on a two-processor machine, where each processor is Intel(R) Xeon(R) Gold 6230 and each processor has 20 cores. We explicitly disabled all cores in node 1 but only utilizing 16 hardware cores in node 0. This machine has 256GB of main memory, 64KB L1 cache, and 1MB of L2 cache. The experimental results are further listed in Table~\ref{tab:independent}. For simplicity, we only listed the applications, the issue number, and serious scores in two different machines. 

Table~\ref{tab:independent} shows that most reported scores in two machines are very similar, although with small variance. The small variance could be caused by multiple factors, such as parallelization degree (concurrency). However, this table shows that all serious issues can be detected on both machines. This indicates that \NP{}  achieves its design goal, which could even detect NUMA issues without running on a NUMA machine. 

% \begin{table}[!htp]
% %\footnotesize
%  	\setlength{\tabcolsep}{0.3em}
% \centering
% \begin{tabular}{|c|l|l|r|r|}
%     \hline
%     \cline{1-5}
%     \multirow{2}{*}{Application}& \multicolumn{4}{c}{Specific Issues}\\
%     \cline{2-5}
%     & \# & Type & \multicolumn{1}{|c|}{\specialcell{Score\\(NUMA)}}  & \multicolumn{1}{|c|}{\specialcell{Score\\(UMA)}}\\ \hline

%     \multirow{2}{*}{AMG2006} & 1 & \PS & & \\
%     \cline{2-5}
    
%     &  2 &\TM&230 &\\ \hline

%     \multirow{7}{*}{lulesh}& 3 &\PS&1978&  \\
%     \cline{2-5}

%     &4&\PS&1611 &  \\
%     \cline{2-5}
    
%     &&5&\PS&1283&\multirow{2}{*}{lulesh.cc:2251-2264}&\BI&406\%&\multirow{2}{*}{418\%}& \\
%     \cline{3-5}\cline{7-8}\cline{10-10}
%     &&6&\FS&174&&\PAD &403\%&& \checkmark\\
%     \cline{3-10}
    
%     &&7&\PS&1316&\multirow{2}{*}{lulesh.cc:2089}&\BI&392\%&\multirow{2}{*}{407\%}& \\
%     \cline{3-5}\cline{7-8}\cline{10-10}
%     &&8&\FS&170&&\PAD &402\%&& \checkmark\\
%     \cline{3-10}
    
%     &&9&\TM&153&&\TB&\multicolumn{2}{|c|}{382\%}&\checkmark \\ \hline
    
%     UMT2013&131\%&10&\TM&253&&\TB&\multicolumn{2}{|c|}{131\%}&\checkmark \\
%     \hline 
%     \hline
    
%     \multirow{3}{*}{bodytrack}&\multirow{3}{*}{109\%}&11&\PS&10444&\multirow{2}{*}{FlexImageStore.h:146}&\PI&\multicolumn{2}{|c|}{\multirow{2}{*}{106\%}}& \\
%     \cline{3-5}\cline{7-7}\cline{10-10}
%     &&12&\FS&328&& &\multicolumn{2}{|c|}{}&\checkmark \\
%     \cline{3-10}
%     &&13&\TM&13646&&\TB&\multicolumn{2}{|c|}{105\%}&\checkmark \\ \hline
    
%     dedup&116\%&14&\TI&&& adjust threads  &\multicolumn{2}{|c|}{116\%}&\checkmark \\ \hline
    
%     facesim&105\%&15&\TM&399&&\TB&\multicolumn{2}{|c|}{105\%}&\checkmark \\ \hline
    
%     ferret&206\%&16&\TI&&& adjust threads  &\multicolumn{2}{|c|}{206\%}&\checkmark \\ \hline
 
%     \multirow{5}{*}{fluidanimate}&\multirow{5}{*}{429\%}&17&\PS&154&\multirow{2}{*}{pthreads.cpp:294}&\PI&112\%&\multirow{2}{*}{160\%}& \\
%     \cline{3-5}\cline{7-8}\cline{10-10}
%     &&18&\FS&223&&\PAD&158\%&&\checkmark \\
%     \cline{3-10}
    
%     &&19&\PS&45994&\multirow{2}{*}{pthreads.cpp:292}& \multirow{2}{*}{\PI} &\multicolumn{2}{|c|}{\multirow{2}{*}{340\%}}& \\
%     \cline{3-5} \cline{10-10}
%     &&20&\TS&19545&&&\multicolumn{2}{|c|}{}&\checkmark \\
%     \cline{3-10}
     
%     &&21&\TM&812&&\TB&\multicolumn{2}{|c|}{418\%}&\checkmark \\ \hline
    
%     \multirow{4}{*}{streamcluster}&\multirow{4}{*}{167\%}&22&\PS&405&\multirow{2}{*}{streamcluster.cpp:984}&\PI&100\%&\multirow{2}{*}{103\%}& \\
%     \cline{3-5}\cline{7-8}\cline{10-10}
%     &&23&\FS&368&&\PAD&102\%&& \checkmark\\
%     \cline{3-10}
     
%     &&24&\PS&6397&streamcluster.cpp:1845&\DUP&\multicolumn{2}{|c|}{158\%}& \\
%     \cline{3-10}
     
%     &&25&\TM&252&&\TB&\multicolumn{2}{|c|}{132\%}&\checkmark \\ \hline
%     \end{tabular}
%   \caption{Detected NUMA performance issues of \NP{}, where it detects 15 performance bugs that cannot be detected using existing NUMA profilers (with a check mark in the last column).}
%   \label{tab:independent}
% \end{table}

%\todo{Can not run UMT2013 successfully in UMA after a lot of efforts, need more time to fix}

\begin{table}[!htp]
%\footnotesize
 	\setlength{\tabcolsep}{0.45em}
\centering
\begin{tabular}{|c|l|l|l|l|}
\hline
\multirow{2}{*}{Application}  & \multicolumn{4}{c|}{Specific   Issues}                                                              \\ \cline{2-5} 
                              & \# & \multicolumn{1}{c|}{Type} & \multicolumn{1}{c|}{\specialcell{Score\\(NUMA)}} & \multicolumn{1}{c|}{\specialcell{Score\\(UMA)}} \\ \hline
\multirow{2}{*}{AMG2006}       & 1  & \PS     & 7390  & 5405  \\ \cline{2-5} 
                               & 2  & thread migration & 6     & 6      \\ \hline
\multirow{7}{*}{lulesh}        & 3  & \PS     & 1840  & 2443  \\ \cline{2-5} 
                               & 4  & \PS     & 1504  & 2353  \\ \cline{2-5} 
                               & 5  & \PS     & 4496  & 4326  \\ \cline{2-5} 
                               & 6  & false sharing    & 26    & 51   \\ \cline{2-5} 
                               & 7  & \PS     & 1229  & 2136  \\ \cline{2-5} 
                               & 8  & false sharing    & 12    & 27    \\ \cline{2-5} 
                               & 9  & thread migration & 3328  & 5213      \\ \hline
UMT2013                        & 10 & thread migration & 18    &      \\ \hline \hline
\multirow{3}{*}{bodytrack}     & 11 & \PS     & 10800 & 8203  \\ \cline{2-5} 
                               & 12 & false sharing    & 24    & 153   \\ \cline{2-5} 
                               & 13 & thread migration & 297   & 190   \\ \hline 
dedup                          & 14 & thread imbalance &   92:1:3    &  88:4:4     \\ \hline
facesim                        & 15 & thread migration & 607   & 274   \\ \hline
ferret*                          & 16 & thread imbalance &      &       \\ \hline
\multirow{5}{*}{fluidanimate} & 17 & \PS              & 90534                            & 15765 \\ \cline{2-5} 
                               & 18 & true sharing     & 2941  & 1753  \\ \cline{2-5} 
                               & 19 & \PS     & 180   & 95   \\ \cline{2-5} 
                               & 20 & false sharing    & 20    & 80    \\ \cline{2-5} 
                               & 21 & thread migration & 73    & 34    \\ \hline
\multirow{4}{*}{streamcluster} & 22 & \PS     & 427   & 270   \\ \cline{2-5} 
                               & 23 & false sharing    & 31    & 153   \\ \cline{2-5} 
                               & 24 & \PS     & 7169  & 10259 \\ \cline{2-5} 
                               & 25 & thread migration & 229   & 214   \\ \hline
\end{tabular}
  \caption{Evaluation on architecture Sensitiveness. We evaluated \NP{} on a non-NUMA (UMA) machine, which has very similar results as that on a NUMA machine. For \texttt{ferret}, \NP{} reports a proportion of  $3:2:48:75$ on the 8-node NUMA machine, and $5:4:50:77$ on the UMA machine. \label{tab:independent}}
  \vspace{-0.2in}
\end{table}




